% Options for packages loaded elsewhere
\PassOptionsToPackage{unicode}{hyperref}
\PassOptionsToPackage{hyphens}{url}
%
\documentclass[
]{article}
\usepackage{amsmath,amssymb}
\usepackage{lmodern}
\usepackage{iftex}
\ifPDFTeX
  \usepackage[T1]{fontenc}
  \usepackage[utf8]{inputenc}
  \usepackage{textcomp} % provide euro and other symbols
\else % if luatex or xetex
  \usepackage{unicode-math}
  \defaultfontfeatures{Scale=MatchLowercase}
  \defaultfontfeatures[\rmfamily]{Ligatures=TeX,Scale=1}
\fi
% Use upquote if available, for straight quotes in verbatim environments
\IfFileExists{upquote.sty}{\usepackage{upquote}}{}
\IfFileExists{microtype.sty}{% use microtype if available
  \usepackage[]{microtype}
  \UseMicrotypeSet[protrusion]{basicmath} % disable protrusion for tt fonts
}{}
\makeatletter
\@ifundefined{KOMAClassName}{% if non-KOMA class
  \IfFileExists{parskip.sty}{%
    \usepackage{parskip}
  }{% else
    \setlength{\parindent}{0pt}
    \setlength{\parskip}{6pt plus 2pt minus 1pt}}
}{% if KOMA class
  \KOMAoptions{parskip=half}}
\makeatother
\usepackage{xcolor}
\usepackage[margin=1in]{geometry}
\usepackage{longtable,booktabs,array}
\usepackage{calc} % for calculating minipage widths
% Correct order of tables after \paragraph or \subparagraph
\usepackage{etoolbox}
\makeatletter
\patchcmd\longtable{\par}{\if@noskipsec\mbox{}\fi\par}{}{}
\makeatother
% Allow footnotes in longtable head/foot
\IfFileExists{footnotehyper.sty}{\usepackage{footnotehyper}}{\usepackage{footnote}}
\makesavenoteenv{longtable}
\usepackage{graphicx}
\makeatletter
\def\maxwidth{\ifdim\Gin@nat@width>\linewidth\linewidth\else\Gin@nat@width\fi}
\def\maxheight{\ifdim\Gin@nat@height>\textheight\textheight\else\Gin@nat@height\fi}
\makeatother
% Scale images if necessary, so that they will not overflow the page
% margins by default, and it is still possible to overwrite the defaults
% using explicit options in \includegraphics[width, height, ...]{}
\setkeys{Gin}{width=\maxwidth,height=\maxheight,keepaspectratio}
% Set default figure placement to htbp
\makeatletter
\def\fps@figure{htbp}
\makeatother
\setlength{\emergencystretch}{3em} % prevent overfull lines
\providecommand{\tightlist}{%
  \setlength{\itemsep}{0pt}\setlength{\parskip}{0pt}}
\setcounter{secnumdepth}{-\maxdimen} % remove section numbering
\ifLuaTeX
  \usepackage{selnolig}  % disable illegal ligatures
\fi
\IfFileExists{bookmark.sty}{\usepackage{bookmark}}{\usepackage{hyperref}}
\IfFileExists{xurl.sty}{\usepackage{xurl}}{} % add URL line breaks if available
\urlstyle{same} % disable monospaced font for URLs
\hypersetup{
  hidelinks,
  pdfcreator={LaTeX via pandoc}}

\author{}
\date{\vspace{-2.5em}}

\begin{document}

{
\setcounter{tocdepth}{2}
\tableofcontents
}

\hypertarget{aim}{%
\subsubsection{Aim}\label{aim}}

The aim of this course is to give an introduction to using R programming
language for analyzing data in RStudio, focusing on basic statistics and
data visualization. It is assumed that you understand data types and be
able to draw conclusions from data results. The goal here is to show you
how to use R to visualize data and teach you some basic statistics. We
will not be explaining in details the suitability of data types to
various statistics and the types of hypothesis that can be tested with
the different statistics.

\hypertarget{overview}{%
\subsubsection{Overview}\label{overview}}

R is a very competent programming language, it is a free, open-sourced
software for statistical analysis, with many useful features such as
``Best-in-Class Visualizations'', that are insightful and facilitate
reproducible research. The accelerated increase in the amounts of data
generated from high-throughput scientific experiments has led to the
introduction of reliable tools such as R, for data processing,
management, and visualization, and with it came the need for people in
academia and industry to upskill themselves with R programming language
for manipulating and visualizing data. Thereby, making R an excellent
programming language to consider for data analysis.

At the end of this training, you should feel confident to start managing
your own data, use packages and libraries in R, produce histograms,
customize plots, build and fit models and perform statistical analyses

\hypertarget{training-schedule}{%
\subsubsection{Training Schedule}\label{training-schedule}}

This course has been parted into two. The first section focuses on data
visualization and the other one is mainly on basic statistics.

\begin{longtable}[]{@{}
  >{\raggedright\arraybackslash}p{(\columnwidth - 6\tabcolsep) * \real{0.4211}}
  >{\raggedright\arraybackslash}p{(\columnwidth - 6\tabcolsep) * \real{0.2105}}
  >{\raggedright\arraybackslash}p{(\columnwidth - 6\tabcolsep) * \real{0.2105}}
  >{\raggedright\arraybackslash}p{(\columnwidth - 6\tabcolsep) * \real{0.1579}}@{}}
\toprule()
\begin{minipage}[b]{\linewidth}\raggedright
Content
\end{minipage} & \begin{minipage}[b]{\linewidth}\raggedright
Date
\end{minipage} & \begin{minipage}[b]{\linewidth}\raggedright
Time
\end{minipage} & \begin{minipage}[b]{\linewidth}\raggedright
Duration
\end{minipage} \\
\midrule()
\endhead
& & & \\
R for Statistics and Data Visualisation - An Introduction for Life
Scientists 1 & & & \\
\href{https://lizzydapsy.github.io/R-course_materials/Getting-started-with-R.html}{Setting
up R and RStudio / Getting started with R} & Mon 13/02/2023 & 10:00:00 -
10:30:00 & 30 minutes \\
\href{https://lizzydapsy.github.io/R-course_materials/Data-exploration-and-basic-plotting.html}{Data
exploration and basic plotting} & Mon 13/02/2023 & 10:30:00 - 11:30:00 &
1 Hour \\
Break & Mon 13/02/2023 & 11:30:00 - 11:45:00 & 15 minutes \\
\href{https://lizzydapsy.github.io/R-course_materials/Data-manipulation-and-advanced-plotting.html}{Data
manipulation and advanced plotting} & Mon 13/02/2023 & 11:45:00 -
13:00:00 & 1.25 Hours \\
& & & \\
R for Statistics and Data Visualisation - An Introduction for Life
Scientists 2 & & & \\
\href{https://lizzydapsy.github.io/R-course_materials/Basic-statistics---t-test.html}{Basic
statistics - t-test} & TBD & 10:00:00 - 11:00:00 & 1 Hour \\
\href{https://lizzydapsy.github.io/R-course_materials/Basic-statistics---chi-square-test.html}{Basic
statistics - chi-square test} & TBD & 11:00:00 - 12:00:00 & 1 Hour \\
Break & TBD & 12:00:00 - 12:30:00 & 30 minutes \\
\href{https://lizzydapsy.github.io/R-course_materials/Linear-Models-and-GLMs.html}{Linear
Models and GLMs} & TBD & 12:30:00 - 13:30:00 & 1 Hour \\
\href{https://lizzydapsy.github.io/R-course_materials/One-way-ANOVA-and-ANCOVA.html}{One-way
ANOVA and ANCOVA} & TBD & 13:30:00 - 14:30:00 & 1 Hour \\
\bottomrule()
\end{longtable}

\hypertarget{general-notes}{%
\subsubsection{General notes}\label{general-notes}}

Here are some references for further reading if needed. You might want
to bookmark them for ease of access

Loads of learning resources:
\url{https://stats.stackexchange.com/questions/138/free-resources-for-learning-r}

R cheatsheet (it's not cheating!):
\url{https://posit.co/resources/cheatsheets/}

Data Carpentry Spreadsheet Formatting Lessons:
\url{https://datacarpentry.org/spreadsheet-ecology-lesson/}

Loads of datasets for practicing: \url{https://www.kaggle.com/datasets}

Tidyverse (ggplot2 and dplyr): \url{https://www.tidyverse.org/}

You are stranded and you need to ask questions:
\url{https://stackoverflow.com/questions/tagged/r}

Any more issues? just google it: \url{https://www.google.com/}

\hypertarget{prerequisites}{%
\subsubsection{Prerequisites}\label{prerequisites}}

No prior programming experience is required

.

\end{document}
